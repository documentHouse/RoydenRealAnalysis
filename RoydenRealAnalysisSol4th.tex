%\documentclass[11pt,reqno]{amsart}
\documentclass[11pt,reqno]{article}
\usepackage[margin=.8in, paperwidth=8.5in, paperheight=11in]{geometry}
%\usepackage{geometry}                % See geometry.pdf to learn the layout options. There are lots.
%\geometry{letterpaper}                   % ... or a4paper or a5paper or ... 
%\geometry{landscape}                % Activate for for rotated page geometry
%\usepackage[parfill]{parskip}    % Activate to begin paragraphs with an empty line rather than an indent7
\usepackage{graphicx}
\usepackage{pstricks}
\usepackage{amssymb}
\usepackage{epstopdf}
\usepackage{amsmath}
\pagestyle{plain}
%\renewcommand{\topfraction}{0.3}
%\renewcommand{\bottomfraction}{0.8}
%\renewcommand{\textfraction}{0.07}
\DeclareGraphicsRule{.tif}{png}{.png}{`convert #1 `dirname #1`/`basename #1 .tif`.png}

\title{Real Analysis - H.L. Royden: \\  4th Edition Solutions}
\author{Andrew Rickert}
\date{Started: Dec 6, 2011}                                           % Activate to display a given date or no date

\begin{document}
\maketitle


% Old Problem Delimiter
%\begin{flushleft} 
%\textbf{Class 18.100B} - Problem 1\\
%\rule{500pt}{1pt}\\
%\end{flushleft} 

\begin{flushleft} 
Problem 1 - Section 2.1, page 31\\
\rule{500pt}{1pt}\\
\end{flushleft} 

Prove that if $A$ and $B$ are two sets in $\mathcal{A}$ with $A \subseteq B$, then $m(A) \le m(B)$. This property is called \emph{monotonicity}.
\\\\ \emph{Proof.}

Let $B' = B \sim A$ so that $B = B' \cap A = \emptyset$ by the construction of the set $B'$. Since $A \subseteq B$ we have that $B = B' \cup A$. Because $A$ and $B'$ are disjoint we use the countable additivity of the measure and that $m(E) \ge 0$ for all $E$ to derive:
\begin{eqnarray*}
m(B) &=& m(A \cup B') = m(A) + m(B') \\
        &&\hspace{53px}\ge m(A)
\end{eqnarray*}

\begin{flushleft} 
Problem 2 - Section 2.1, page 31\\
\rule{500pt}{1pt}\\
\end{flushleft} 

Prove that if there is a set $A$ in the collection $\mathcal{A}$ for which $m(A) < \infty$, then $m(\emptyset) = 0$.
\\\\ \emph{Proof.}

Let $A$ have finite measure. Note that since $A \cap \emptyset = \emptyset$ that $A$ and $\emptyset$ are disjoint. We may then apply countable additivity to show:
\begin{equation*}
m(A) = m(A \cup \emptyset) = m(A) + m(\emptyset) \implies m(\emptyset) = 0
\end{equation*}


\begin{flushleft} 
Problem 2 - Section 2.1, page 31\\
\rule{500pt}{1pt}\\
\end{flushleft} 

Let $\{ E_k \}^\infty_{k = 1}$ be a countable collection of sets in $\mathcal{A}$. Prove that $m(\cup^\infty_{k=1} E_k) \le \sum^\infty_{k = 1} m(E_k)$
\\\\ \emph{Proof.}

\end{document}  
