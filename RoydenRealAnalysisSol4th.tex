%\documentclass[11pt,reqno]{amsart}
\documentclass[11pt,reqno]{article}
\usepackage[margin=.8in, paperwidth=8.5in, paperheight=11in]{geometry}
%\usepackage{geometry}                % See geometry.pdf to learn the layout options. There are lots.
%\geometry{letterpaper}                   % ... or a4paper or a5paper or ... 
%\geometry{landscape}                % Activate for for rotated page geometry
%\usepackage[parfill]{parskip}    % Activate to begin paragraphs with an empty line rather than an indent7
\usepackage{graphicx}
\usepackage{pstricks}
\usepackage{amssymb}
\usepackage{epstopdf}
\usepackage{amsmath}
\pagestyle{plain}
%\renewcommand{\topfraction}{0.3}
%\renewcommand{\bottomfraction}{0.8}
%\renewcommand{\textfraction}{0.07}
\DeclareGraphicsRule{.tif}{png}{.png}{`convert #1 `dirname #1`/`basename #1 .tif`.png}

\title{Real Analysis - H.L. Royden: \\  4th Edition Solutions}
\author{Andrew Rickert}
\date{Started: Dec 6, 2011}                                           % Activate to display a given date or no date

\begin{document}
\maketitle


% Old Problem Delimiter
%\begin{flushleft} 
%\textbf{Class 18.100B} - Problem 1\\
%\rule{500pt}{1pt}\\
%\end{flushleft} 

\begin{flushleft} 
Problem 1 - Section 2.1, page 31\\
\rule{500pt}{1pt}\\
\end{flushleft} 

Prove that if $A$ and $B$ are two sets in $\mathcal{A}$ with $A \subseteq B$, then $m(A) \le m(B)$. This property is called \emph{monotonicity}.
\\\\ \emph{Proof.}

Let $B' = B \sim A$ so that $B = B' \cap A = \emptyset$ by the construction of the set $B'$. Since $A \subseteq B$ we have that $B = B' \cup A$. Because $A$ and $B'$ are disjoint we use the countable additivity of the measure and that $m(E) \ge 0$ for all $E$ (because we are given that $m: \mathcal{A} \to [0,\infty]$)to derive:
\begin{eqnarray*}
m(B) &=& m(A \cup B') = m(A) + m(B') \\
        &&\hspace{53px}\ge m(A)
\end{eqnarray*}

\begin{flushleft} 
Problem 2 - Section 2.1, page 31\\
\rule{500pt}{1pt}\\
\end{flushleft} 

Prove that if there is a set $A$ in the collection $\mathcal{A}$ for which $m(A) < \infty$, then $m(\emptyset) = 0$.
\\\\ \emph{Proof.}

Let $A$ have finite measure. Note that since $A \cap \emptyset = \emptyset$ that $A$ and $\emptyset$ are disjoint. We may then apply countable additivity to show:
\begin{equation*}
m(A) = m(A \cup \emptyset) = m(A) + m(\emptyset) \implies m(\emptyset) = 0
\end{equation*}


\begin{flushleft} 
Problem 3 - Section 2.1, page 31\\
\rule{500pt}{1pt}\\
\end{flushleft} 

Let $\{ E_k \}^\infty_{k = 1}$ be a countable collection of sets in $\mathcal{A}$. Prove that $m(\cup^\infty_{k=1} E_k) \le \sum^\infty_{k = 1} m(E_k)$
\\\\ \emph{Proof.}

First we show  $m(\cup^n_{k=1} E_k) \le \sum^n_{k = 1} m(E_k)$ by induction.\\
Let $n = 2$ so that we first want to show $m(A \cup B) \le m(A) + m(B)$. Let $O = A \cap B$ and let $A' = A \sim O$ and $B' = B \sim O$. It is clear from construction that we have
\begin{align}
&A \cup B = A' \cup O \cup B' \label{unionbreak} \\
&A = A' \cup O \hspace{15px} \text{and} \hspace{15px} B = B' \cup O \label{intersum} \\
&(A' \cap O) \cap B' = \emptyset \label{uniondisjoint} \\
&A' \subset A \hspace{15px} \text{and} \hspace{15px} B' \subset B \label{partsubs}
\end{align}

\noindent We may use the countable additivity of $m$ to derive:
\begin{eqnarray*}
m(A \cup B) = m(A' \cup O \cup B') &=& m(A' \cup O) + m(B') \hspace{20px} \text{By $(\ref{unionbreak})$ and $(\ref{uniondisjoint})$} \\		  
 &=&  m(A) + m(B') \hspace{20px} \text{By $(\ref{intersum})$} \\
 &\le&  m(A) + m(B') \hspace{20px} \text{By $(\ref{partsubs}$) and the monotonocity shown in the previous problem}
\end{eqnarray*}

\noindent Now we show the induction step.
\begin{eqnarray*}
m(\cup^{n+1}_{k=1} E_k) &=& m((\cup^n_{k=1} E_k) \cup E_{n+1}) \\
&\le& m(\cup^n_{k=1} E_k) + m(E_{n+1}) \hspace{73px} \text{By the induction base} \\
&\le&  \sum^n_{k = 1} m(E_k) + m(E_{n+1}) =  \sum^{n+1}_{k = 1} m(E_k) \hspace{15px} \text{By the induction step} \\
\end{eqnarray*}

So we have to show that $m(\cup^n_{k=1} E_k) \le \sum^n_{k = 1} m(E_k)$. Since for all $E$ we have $0 \le m( E)$ it is true for all $n$ that
\[ \sum^n_{k=1} m(E_k) \le \sum^\infty_{k = 1} m(E_k) \]
By the previous result we have 
\[ m(\cup^{n}_{k=1} E_k) \le \sum^n_{k=1} m(E_k) \le \sum^\infty_{k = 1} m(E_k) \implies m(\cup^{n}_{k=1} E_k) \le \sum^\infty_{k = 1} m(E_k)\]
Since this expression is independent of $n$ we make take the limit to get 
\[\lim_{n \to \infty} m(\cup^{n}_{k=1} E_k) = m(\cup^{\infty}_{k=1} E_k) \le \sum^\infty_{k = 1} m(E_k) \]


\begin{flushleft} 
Problem 4 - Section 2.1, page 31\\
\rule{500pt}{1pt}\\
\end{flushleft} 

A set function $c$, defined on all subsets of $\mathbb{R}$, is defined as follows. Define $c(E)$ to be $\infty$ if $E$ has infinityle many members and $c(E)$ to be equal to the number of elements in $E$ if $E$ is finite; define $c(\emptyset) = 0$. Show that $c$ is a countable additive and translation invariant set function. This set function is called the \textbf{counting measure}.
\\\\ \emph{Proof.}

We first show the translation invariance of $c$. \\
First we note that the translation of $\emptyset$ by $y \in \mathbb{R}$ is still $\emptyset$ so $c(\emptyset + y) = c(\emptyset) = 0$. Now let $f$ be a function such that $f: E \to E + y$ defined as follows. If $x \in E$ then $f$ takes $x \to x + y \in E + y$. This function is clearly bijective so $E$ and $E + y$ have the same 'number' of elements. That is, finite sets are taken to finite sets and infinite sets are taken to infinite sets. This means that $c(E) = c(E + y)$ which finishes the translation invariance of $c$.\\
\indent Now we show the countable additivity of $c$.
We consider $\{ E_k \}^\infty_{k = 1}$, a disjoint countable collection of sets.\\
If any of the sets $E_k$ is infinite then the relationship 
\begin{equation}
c(\cup^\infty_{k = 1} E_k) = \sum^\infty_{k = 1} c(E_k) \label{countingmeas-ca}
\end{equation}
 is trivially true. We now assume that all $E_k$ are finite.\\
\indent First we assume that there are infinitly many $E_k$ that are non-empty and we drop the empty sets from the equation since they don't contribute to the union or the sum in (\ref{countingmeas-ca}).
Since there are an infinite number of $E_k$, all of which are disjoint, we have $c(\cup^\infty_{k = 1} E_k) = \infty$. This is so because we may choose a member from each $E_k$ which is distinct from the other members of the sets. We can do this for each $k$ so we form a new set $E'$ which is the union of these elements and is 1 to 1 correspondence with $\mathbb{N}$. This is a subset of $(\cup^\infty_{k = 1} E_k$ so we have the state counting measure.\\
\indent Also, because we know there are an infinity of $E_k$ which are nonempty we know that $1 \le c(E_k) \implies \infty \le c(\sum_{k = 1}^\infty E_k) \implies c(\sum_{k = 1}^\infty E_k) = \infty $. This once again gives the relation in (\ref{countingmeas-ca}).\\
The case of a finite number of disjoint $E_k$ is a simple induction and proof is complete.


\begin{flushleft} 
Problem 5 - Section 2.2, page 34\\
\rule{500pt}{1pt}\\
\end{flushleft} 

By using the properties of outer measure, prove that the interval $[0,1]$ is not countable.
\\\\ \emph{Proof.}

As was shown in the text, the outer measure of countable sets is 0. One of the properties of outer measure is that the outer measure of an interval is its length. Since $m^*([0,1])= 1 \neq 0$ the assumption that $[0,1]$ is countable leads to a contradiction.

\begin{flushleft} 
Problem 6 - Section 2.2, page 34\\
\rule{500pt}{1pt}\\
\end{flushleft} 

Let $A$ be the set of irrational numbers in the interval $[0,1]$. Prove that $m^*(A) = 1$.
\\\\ \emph{Proof.}

Let $\mathbb{Q}' = [0,1] \cap \mathbb{Q}$ and  $A = [0,1] \cap \mathbb{I}$ where $\mathbb{Q}$ are the rational numbers in $\mathbb{R}$. We note that $[0,1] = \mathbb{Q}' \cup A$ and $\mathbb{Q}' \cap A = \emptyset$.

First, $A' \subset [0,1]$ so we have 
\[ m^*(A) \le m^*([0,1]) = 1\] 
by monotonocity. On the other hand we have 
\[1 = m^*([0,1]) = m^*(\mathbb{Q}' \cup A) \le  m^*(A) +  m^*(\mathbb{Q}')  =  m^*(A)\]
since $\mathbb{Q}'$ is a subset of the rationals which have measure zero.
So we have both $ m^*(A) \le 1$ and $1 \le m^*(A)$  which implies $m^*(A) = 1$.

\begin{flushleft} 
Problem 7 - Section 2.2, page 34\\
\rule{500pt}{1pt}\\
\end{flushleft} 

A set of real numbers is said to be a $G_\delta$ set provided it is the intersection of a countable collection of open sets. Show that for any bounded set $E$, there is a $G_\delta$ set $G$ for which 
\[ E \subset G \; \text{and} \; m^*(G) = m^*(E) \]
\\\\ \emph{Proof.}

The outer measure $m^*(E)$ is defined as the infimum of the sums $\sum_{k = 1}^\infty \ell(I_k)$ where the $I_k$ are intervals covering the set $E$. This means that for each $k$ we can chose a collection of open intervals $I_j$ such that  $m^*(E) \le \sum_{j = 1}^\infty \ell(I_j) \le m^*(E) + 1/2^k$ from the definition of the infimum.\\
\indent Since the $I_j$ are open we can construct a set for each $k$ as $G_k = \cup_j I_j$ which is open since it is a union of open sets. We also know that $E \subset G_k$ for all $k$ because $G_k$ is the open cover formed by the intervals $I_j$. \\
\indent We now form the set $G = \cap_{k=1} G_k$. From the fact that $E \subset G_k$ for all $k$ we immediately get that $E \subset G$.\\
\indent We also take advantage of the property that $A \cap B \subset A$ and $A \cap B \subset B$ to note $G = \cap_{k=1} G_k \subset G_k$ for all $k$. \\
This allows us to derive the final property for the proof
\begin{eqnarray*}
&&m*(E) \le m^*(G) \le m^*(G_k) =  \sum_{j = 1}^\infty \ell(I_j) \le m^*(E) + 1/2^k\\
&& \implies m^*(E) \le m^*(G) \le m^*(E) + 1/2^k \, \text{for all $k$} \\
&& \implies m^*(E) = m^*(G)\\
\end{eqnarray*}


\begin{flushleft} 
Problem 8 - Section 2.2, page 34\\
\rule{500pt}{1pt}\\
\end{flushleft} 

Let $B$ be the set of rational numbers in the interval [0,1], and let $\{ I_k \}^n_{k=1}$ be a finite collection of open intervals that covers $B$. Prove that $\sum^n_{k=1} m^*(I_k) \ge 1$.
\\\\ \emph{Proof.}

It is true that if $\overline{B}$ is the closure of $B$ and $ B \subset C = \cup_{k = 1}^n I_k$ then $\overline{B} \subset \overline{C}$. We also know that $\overline{ \cup_{k = 1}^n I_k} = \cup_{k = 1}^n \overline{I_k}$ which in this case means the open intervals $I_k$ become closed intervals $\overline{I_k}$. Since $\overline{B} = [0,1]$ we have that the closed intervals $\{\overline{I_k}\}$ cover $[0,1]$.\\
\indent The open and closed intervals of the same length have the same the outer measure so this allows us to show:
\[ \sum_{k = 1}^n m^*(I_k) =  \sum_{k = 1}^n m^*(\overline{I_k}) \ge m^*(\cup_{k = 1}^n \overline{I_k}) \ge m^*([0,1]) = 1 \]

\begin{flushleft} 
Problem 9 - Section 2.2, page 34\\
\rule{500pt}{1pt}\\
\end{flushleft} 

Prove that if $m^*(A) = 0$, then $m^*(A \cup B) = m^*(B)$
\\\\ \emph{Proof.}

From the fact that $B \subset A \cup B$ we have from monotonocity that $m^*(B) \le m^*(A \cup B)$. We also have by countable subadditivity that $m^*(A \cup B) \le m^*(A) + m^*(B) = m^*(B)$. These two relationships give $m^*(B) = m^*(A \cup B)$.

\begin{flushleft} 
Problem 10 - Section 2.2, page 34\\
\rule{500pt}{1pt}\\
\end{flushleft} 

Let $A$ and $B$ be bounded sets for which there is an $\alpha > 0$ such that $|a - b| \ge \alpha$ for all $a \in A, b \in B$. Prove that $m^*(A \cup B) = m^*(A) + m ^*(B)$.
\\\\ \emph{Proof.}

By countable additivity we know that $m^*(A \cup B) \le m^*(A) + m^*(B)$ so it only remains to show that $m^*(A \cup B) \ge m^*(A) + m^*(B)$.

\begin{flushleft} 
Problem 11 - Section 2.2, page 39\\
\rule{500pt}{1pt}\\
\end{flushleft} 

Prove that if a $\sigma$-algebra of subsets of $\mathbb{R}$ contains intervals of the form $(a,\infty)$, then it contains all intervals.
\\\\ \emph{Proof.}

We make take the complement of $(a,\infty)$ with respect to $\mathbb{R}$, since we have a $\sigma$-algebra, to get all intervals of the form $(-\infty, b]$.\\
We may then also derive the following intervals because countable unions and intersections are again in the $\sigma$-algebra.

\begin{eqnarray*}
(-\infty, b) &=& \cup_{k = 1}^\infty (-\infty, b - \frac{1}{n}] \\
\left[a, \infty \right) &=& \cap_{k = 1}^\infty (a - \frac{1}{n}, \infty]
\end{eqnarray*}

We may now derive the final intervals as follows, again appealing to the fact that we are in a $\sigma$-algebra.

\begin{eqnarray*}
(a, b) &=& (a, \infty) \cap (-\infty, b) \\
(a, b] &=& (a, \infty) \cap (-\infty, b] \\
\left[a, b\right) &=& [a, \infty) \cap (-\infty, b) \\
\left[a, b\right] &=& [a, \infty) \cap (-\infty, b] \\
\end{eqnarray*}


\begin{flushleft} 
Problem 12 - Section 2.2, page 39\\
\rule{500pt}{1pt}\\
\end{flushleft} 

Show that every interval is a Borel set.
\\\\ \emph{Proof.}

The Borel set is the $\sigma$ algebra that contains all the open sets. Since $(a, \infty)$ is an open set the result of Problem 11 shows that the Borel sets contain all intervals.

\begin{flushleft} 
Problem 13 - Section 2.2, page 40\\
\rule{500pt}{1pt}\\
\end{flushleft} 

Show that (i) the translate of an $F_\sigma$ set is also $F_\sigma$, (ii) the translate of a $G_\delta$ set is also $G_\delta$ , and (iii) the translate of a set of measure zero also has measure zero.
\\\\ \emph{Proof.}

We start by showing that the translate of an open set is open. We note first that the translate of open interval $(a,b)$ is $(a + y, b + y)$ which is still an open interval. An open set is a countable union of open intervals so we know that 
\[ O = \cup^\infty_{n = 1}(a_n, b_n) \implies  O + y = \cup^\infty_{n = 1}(a_n + y, b_n + y)  \]

Since each of the intervals $(a_n + y, b_n + y)$ is open by the previous comment we have $O +  y$ as a countable union of open sets which is open.\\
\indent We also need to show that the translation of a closed set is also closed. We do this by first noting that if $C$ is a closed set and the complement of $C$ in $\mathbb{R}$ is $C'$ then we have the following relationship $C' + y = (C + y)'$ since
\[ x \in C' + y \iff x - y \in C' \iff x- y \notin C \iff x \notin C + y \iff x \in (C + y)' \] \\
Now since $C$ is closed we know that $C'$ is open. We form the translate and use the previous relationship to note that since $C' + y = (C + y)'$ that $(C + y)'$ must be open. This means that $C+y$ must be closed so we have shown that 
\[ C \; \text{is closed} \implies C + y \; \text{is also closed} \]


\noindent We can solve the parts of the problem:\\

\noindent (i) Let $F \in F_\sigma$ so that $F = \cup^\infty_{n = 1} C_n$ where $C_n$ are closed sets. We know that $F + y = \cup^\infty_{n = 1} C_n + y$ and by the previous comments each of the $C_n + y$ is closed so $F + y$ is a countable intersection of closed sets therefore $F + y \in F_\sigma$\\

\noindent (ii) Similarly, let $G \in G_\delta$ so that $G = \cap^\infty_{n = 1} O_n$ where $O_n$ are open sets. We know that $G + y = \cap^\infty_{n = 1} O_n + y$ and by the previous comments each of the $O_n + y$ is open so $G + y$ is a countable union of open sets therefore $G + y \in G_\delta$\\

\noindent (iii) We first note that by hypothesis the measurable set $E$ is such that $m^*(E) = 0$. By proposition 10 we know that the translate of a measurable set is measurable. By the translation invariance of measure we also know that $m^*(E  +y) = m^*(E) = 0$.

\begin{flushleft} 
Problem 14 - Section 2.2, page 40\\
\rule{500pt}{1pt}\\
\end{flushleft} 

Show that if a set $E$ has positive outer measure, then there is a bounded subset of $E$ that also has positive outer measure.
\\\\ \emph{Proof.}
 
Let $I_n$ be the interval $[n, n + 1] $,  then it is the case that $E = \cup_{n = 1} ^\infty  E\cap I_n $.  From the property of countable subadditivity we have that $0 \le m^*(E) \le \sum_{n  = 1}^\infty m^*(E\cap I_n)$ which shows that one of the intersections of $E$ with $I_n$ must have a positive value. \\
\indent If this were not the case then the outer measure of all of the intersections of $E$ with the intervals would be zero. By the earlier statement that $0 \le m^*(E) \le \sum_{n  = 1}^\infty m^*(E\cap I_n)$ this would imply that the outer measure of $E$ is zero contrary to our assumption. Therefore one of the intersections with the intervals $I_n$ must have a positive outer measure.

\begin{flushleft} 
Problem 15 - Section 2.2, page 40\\
\rule{500pt}{1pt}\\
\end{flushleft} 

Show that if $E$ has finite measure and $\epsilon > 0$, then $E$ is the disjoint union of a finite number of measurable sets, each of which as measure at most $\epsilon$.
\\\\ \emph{Proof.}

It must be the case that there exists an $M$ such that $m^*(E \sim [-M, M]) \le \epsilon$.  This is to say there must be a bounded interval $[-M, M]$ that contains all but $\epsilon$ of the set $E$. If this were not the case then we have  $m^*(E \sim [-M, M]) > \epsilon$ for all $M$ which says that $E$ does not have finite measure. If we let $S = E \sim [-M, M]$ then $m^*(S) \le \epsilon$.\\
\indent We now break up $[-M, M]$ into sets of size at most $\epsilon$ based on the following pattern:

\begin{eqnarray*}
S_1 &=& [-M, -M + \epsilon)\\
S_2 &=& [-M + \epsilon, -M + 2 \epsilon)\\
S_3 &=& [-M + 2 \epsilon, -M + 3 \epsilon]\\
\ldots&&
\end{eqnarray*}

\indent Since we have $S \cup S_1 \cup S_2 \cdots = \mathbb{R}$ we know that $E$ is a subset of this union each of whose elements are disjoint and of outer measure less than or equal to $\epsilon$. The disjoint union we are looking for is therefore
\[ E = (E \cap S) \cup (E \cap S_1) \cup (E \cap S_2) \cdots \]


%\begin{\eqnarray*}
%S_1 &=& [-M, -M + \epsilon]\\
%S_1 &=& [-M, -M + \epsilon]
%\end{\eqnarray*}


\end{document}  
