%\documentclass[11pt,reqno]{amsart}
\documentclass[11pt,reqno]{article}
\usepackage[margin=.8in, paperwidth=8.5in, paperheight=11in]{geometry}
%\usepackage{geometry}                % See geometry.pdf to learn the layout options. There are lots.
%\geometry{letterpaper}                   % ... or a4paper or a5paper or ... 
%\geometry{landscape}                % Activate for for rotated page geometry
%\usepackage[parfill]{parskip}    % Activate to begin paragraphs with an empty line rather than an indent7
\usepackage{graphicx}
\usepackage{pstricks}
\usepackage{amssymb}
\usepackage{epstopdf}
\usepackage{amsmath}
\pagestyle{plain}
%\renewcommand{\topfraction}{0.3}
%\renewcommand{\bottomfraction}{0.8}
%\renewcommand{\textfraction}{0.07}
\DeclareGraphicsRule{.tif}{png}{.png}{`convert #1 `dirname #1`/`basename #1 .tif`.png}

\title{Real Analysis - H.L. Royden: \\  4th Edition Solutions}
\author{Andrew Rickert}
\date{Started: Dec 6, 2011}                                           % Activate to display a given date or no date

\begin{document}
\maketitle


% Old Problem Delimiter
%\begin{flushleft} 
%\textbf{Class 18.100B} - Problem 1\\
%\rule{500pt}{1pt}\\
%\end{flushleft} 

\begin{flushleft} 
Problem 1 - Section 2.1, page 31\\
\rule{500pt}{1pt}\\
\end{flushleft} 

Prove that if $A$ and $B$ are two sets in $\mathcal{A}$ with $A \subseteq B$, then $m(A) \le m(B)$. This property is called \emph{monotonicity}.
\\\\ \emph{Proof.}

Let $B' = B \sim A$ so that $B = B' \cap A = \emptyset$ by the construction of the set $B'$. Since $A \subseteq B$ we have that $B = B' \cup A$. Because $A$ and $B'$ are disjoint we use the countable additivity of the measure and that $m(E) \ge 0$ for all $E$ (because we are given that $m: \mathcal{A} \to [0,\infty]$)to derive:
\begin{eqnarray*}
m(B) &=& m(A \cup B') = m(A) + m(B') \\
        &&\hspace{53px}\ge m(A)
\end{eqnarray*}

\begin{flushleft} 
Problem 2 - Section 2.1, page 31\\
\rule{500pt}{1pt}\\
\end{flushleft} 

Prove that if there is a set $A$ in the collection $\mathcal{A}$ for which $m(A) < \infty$, then $m(\emptyset) = 0$.
\\\\ \emph{Proof.}

Let $A$ have finite measure. Note that since $A \cap \emptyset = \emptyset$ that $A$ and $\emptyset$ are disjoint. We may then apply countable additivity to show:
\begin{equation*}
m(A) = m(A \cup \emptyset) = m(A) + m(\emptyset) \implies m(\emptyset) = 0
\end{equation*}


\begin{flushleft} 
Problem 3 - Section 2.1, page 31\\
\rule{500pt}{1pt}\\
\end{flushleft} 

Let $\{ E_k \}^\infty_{k = 1}$ be a countable collection of sets in $\mathcal{A}$. Prove that $m(\cup^\infty_{k=1} E_k) \le \sum^\infty_{k = 1} m(E_k)$
\\\\ \emph{Proof.}

First we show  $m(\cup^n_{k=1} E_k) \le \sum^n_{k = 1} m(E_k)$ by induction.\\
Let $n = 2$ so that we first want to show $m(A \cup B) \le m(A) + m(B)$. Let $O = A \cap B$ and let $A' = A \sim O$ and $B' = B \sim O$. It is clear from construction that we have
\begin{align}
&A \cup B = A' \cup O \cup B' \label{unionbreak} \\
&A = A' \cup O \hspace{15px} \text{and} \hspace{15px} B = B' \cup O \label{intersum} \\
&(A' \cap O) \cap B' = \emptyset \label{uniondisjoint} \\
&A' \subset A \hspace{15px} \text{and} \hspace{15px} B' \subset B \label{partsubs}
\end{align}

\noindent We may use the countable additivity of $m$ to derive:
\begin{eqnarray*}
m(A \cup B) = m(A' \cup O \cup B') &=& m(A' \cup O) + m(B') \hspace{20px} \text{By $(\ref{unionbreak})$ and $(\ref{uniondisjoint})$} \\		  
 &=&  m(A) + m(B') \hspace{20px} \text{By $(\ref{intersum})$} \\
 &\le&  m(A) + m(B') \hspace{20px} \text{By $(\ref{partsubs}$) and the monotonocity shown in the previous problem}
\end{eqnarray*}

\noindent Now we show the induction step.
\begin{eqnarray*}
m(\cup^{n+1}_{k=1} E_k) &=& m((\cup^n_{k=1} E_k) \cup E_{n+1}) \\
&\le& m(\cup^n_{k=1} E_k) + m(E_{n+1}) \hspace{73px} \text{By the induction base} \\
&\le&  \sum^n_{k = 1} m(E_k) + m(E_{n+1}) =  \sum^{n+1}_{k = 1} m(E_k) \hspace{15px} \text{By the induction step} \\
\end{eqnarray*}

So we have to show that $m(\cup^n_{k=1} E_k) \le \sum^n_{k = 1} m(E_k)$. Since for all $E$ we have $0 \le m( E)$ it is true for all $n$ that
\[ \sum^n_{k=1} m(E_k) \le \sum^\infty_{k = 1} m(E_k) \]
By the previous result we have 
\[ m(\cup^{n}_{k=1} E_k) \le \sum^n_{k=1} m(E_k) \le \sum^\infty_{k = 1} m(E_k) \implies m(\cup^{n}_{k=1} E_k) \le \sum^\infty_{k = 1} m(E_k)\]
Since this expression is independent of $n$ we make take the limit to get 
\[\lim_{n \to \infty} m(\cup^{n}_{k=1} E_k) = m(\cup^{\infty}_{k=1} E_k) \le \sum^\infty_{k = 1} m(E_k) \]


\begin{flushleft} 
Problem 4 - Section 2.1, page 31\\
\rule{500pt}{1pt}\\
\end{flushleft} 

A set function $c$, defined on all subsets of $\mathbb{R}$, is defined as follows. Define $c(E)$ to be $\infty$ if $E$ has infinityle many members and $c(E)$ to be equal to the number of elements in $E$ if $E$ is finite; define $c(\emptyset) = 0$. Show that $c$ is a countable additive and translation invariant set function. This set function is called the \textbf{counting measure}.
\\\\ \emph{Proof.}

We first show the translation invariance of $c$. \\
First we note that the translation of $\emptyset$ by $y \in \mathbb{R}$ is still $\emptyset$ so $c(\emptyset + y) = c(\emptyset) = 0$. Now let $f$ be a function such that $f: E \to E + y$ defined as follows. If $x \in E$ then $f$ takes $x \to x + y \in E + y$. This function is clearly bijective so $E$ and $E + y$ have the same 'number' of elements. That is, finite sets are taken to finite sets and infinite sets are taken to infinite sets. This means that $c(E) = c(E + y)$ which finishes the translation invariance of $c$.\\
\indent Now we show the countable additivity of $c$.
We consider $\{ E_k \}^\infty_{k = 1}$, a disjoint countable collection of sets.\\
If any of the sets $E_k$ is infinite then the relationship 
\begin{equation}
c(\cup^\infty_{k = 1} E_k) = \sum^\infty_{k = 1} c(E_k) \label{countingmeas-ca}
\end{equation}
 is trivially true. We now assume that all $E_k$ are finite.
First we assume that there are infinitly many $E_k$ that are non-empty and we drop the empty sets from the equation since they don't contribute to the union or the sum in (\ref{countingmeas-ca}).

\end{document}  
